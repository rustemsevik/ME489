\section{Summary}
In this homework we see that the application of the k-means clustering algorithm can be successfuly used for clustring data. The algorthim can give quite well results with random initializing and it is open to further improvements with proper choice of initialization. In our case we both examine the random initializing and simple method called median based approach, after several tries with given data sets, we see that utilizing median based approach improved the clustering results for our datasets. In this way we have obtained much clearer and more consistent results. \cite{PENA19991027} states that the K-Means clustering algorithm is an iterative algorithm which is sensitive to the initial conditions and it converges finitely to a local minima. These findings highlight the importance of choosing an effective initialization strategy in k-means clustering to enhance the quality of the clustering results.